
% Template for ICIP-2019 paper; to be used with:
%          spconf.sty  - ICASSP/ICIP LaTeX style file, and
%          IEEEbib.bst - IEEE bibliography style file.
% --------------------------------------------------------------------------
\documentclass{article}
\usepackage{spconf,amsmath,graphicx}

% Example definitions.
% --------------------
\def\x{{\mathbf x}}
\def\L{{\cal L}}

% Title.
% ------
\title{TBC}
%
% Single address.
% ---------------
\name{Ross Erskine (ppxre1)}
\address{The University of Nottingham}
%
% For example:
% ------------
%\address{School\\
%	Department\\
%	Address}
%
% Two addresses (uncomment and modify for two-address case).
% ----------------------------------------------------------
%\twoauthors
%  {A. Author-one, B. Author-two\sthanks{Thanks to XYZ agency for funding.}}
%	{School A-B\\
%	Department A-B\\
%	Address A-B}
%  {C. Author-three, D. Author-four\sthanks{The fourth author performed the work
%	while at ...}}
%	{School C-D\\
%	Department C-D\\
%	Address C-D}
%
\begin{document}
%\ninept
%
\maketitle
%
\begin{abstract}
an abstract of report small intro methods used an findings
\end{abstract}
%
\begin{keywords}
Computer vision, Face recognition
\end{keywords}
%
\section{Introduction}
\label{sec:intro}
Introduce project/report


\section{Methodology}
\label{sec:method}
Methods used in project\cite{taigman_deepface_2014}

 \subsection{DataSet}
\label{ssec:data}
DataSet used and why?\cite{wang_deep_2021}


 \subsection{Architecture}
\label{ssec:arch}
Architecture used and eplained individual layers

 \subsection{Feature extraction}
\label{ssec:featex}

 \subsection{Face verification}
\label{ssec:face}

 \subsection{Loss function}
\label{ssec:loss}
Loss function used and why?

\subsection{Normalisation}
\label{ssec:norm}
Normalisation used and why?


\section{Method Evaluation}
\label{sec:methodeval}

Evaluation of methods used

\section{Conclusion}
\label{sec:con}

Conclude with a summary of project/report



% To start a new column (but not a new page) and help balance the last-page
% column length use \vfill\pagebreak.
% -------------------------------------------------------------------------
%\vfill
%\pagebreak




% References should be produced using the bibtex program from suitable
% BiBTeX files (here: strings, refs, manuals). The IEEEbib.bst bibliography
% style file from IEEE produces unsorted bibliography list.
% -------------------------------------------------------------------------
\bibliographystyle{IEEEbib}
\bibliography{CV.bib}

\end{document}
